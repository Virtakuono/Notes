\documentclass[reqno]{amsart} 
\usepackage{amsmath,amssymb,amsthm}
\usepackage{mathabx}
\usepackage[pdftex]{graphicx}
\usepackage{a4wide}
\usepackage{algorithm,algorithmic}
\usepackage{makros}
\usepackage{hyperref}
\usepackage{color}
\usepackage{soul}



\begin{document}

\title[Sums of powers of three]{Sums of powers of three}

\author[J.~H{\"a}pp{\"o}l{\"a}]{Juho H{\"a}pp{\"o}l{\"a}}



\begin{abstract}
  A formula for summing positive powers of three is derived.
   \end{abstract}


\maketitle

Define the difference equation
\begin{align*}
x_{n+1} = 3x_n+1
\end{align*}
then
\begin{align*}
x_{n+2} =& 3x_{n+1}+1, \\
x_{n+1} =& 3x_{n}+1, \\
x_{n+1} -x_{n+2} =& 3x_{n}-3x_{n+1}, \\
x_{n+2}-4 x_{n+1} +3 x_n =& 0 
\end{align*}
take ansatz $x_n=r^n$:
\begin{align*}
r^2 - 4r+3 &= 0 \\
(r-3)(r-1) &= 0
\end{align*}
which gives that the difference equation has a solution
\begin{align*}
x_n = 3^nC +D
\end{align*}
for some constants C and D.
(cf. \url{https://en.wikipedia.org/w/index.php?title=Recurrence_relation&oldid=602802902})
These constants are defined as
\begin{align*}
C + D &= x_0 \\
3C + D &= 3x_0 +1,
\end{align*}
giving us $C = x_0+\frac{1}{2}$, $D =- \frac{1}{2}$
On the other hand, we have
\begin{align*}
x_1 =& 3x_0 + 1\\
x_2 =& 9x_0 + 3 +1\\
x_3 =& 27 x_0 + 9 +3 +1 \\
x_n =& 3^n x_0 + \sum_{j=0}^{n-1} 3^{j}
\end{align*}
giving us that
\begin{align*}
3^n x_0 + \sum_{j=0}^{n-1} 3^{j} &= %3^n x_0 -\frac{3}{2} x_0 - 2^{-1} \\
%\sum_{j=0}^{n-1} 3^{j} &= (x_0 + 2^{-1}) x_0 - 2^{-1} -3^n x_0
3^n x_0 +\frac{3^n-1}{2}
\\
 \sum_{j=0}^{n-1} 3^{j} &= 
 \frac{3^n-1}{2}.
\end{align*}
This derivation perhaps enables one to compute sums of powers of other numbers too, but
I need only powers of three right now.


\end{document}